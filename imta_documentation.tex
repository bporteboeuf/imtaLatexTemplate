\documentclass{article}

\usepackage{imta_core}
\usepackage{imta_extra}

\usepackage[francais]{babel}

\author{FOUCAULT Armand\\PORTEBOEUF Benoît}
\date{December 2017}
\title{Institut Mines-Télécom Atlantique - \LaTeX report template}
\subtitle{Documentation for the \texttt{imta} package}

\imtaSetIMTStyle


%%%%%%%%%%%%%%%%%%%%%%%%%%%%%%% 
%%%%%%%%%% BEGINNING %%%%%%%%%% 
\begin{document}

\imtaMaketitlepage

\tableofcontents

% \chapter{Core features: \texttt{imtacore}}

\section{Sectioning}

\subsection{\texttt{imtaQuestion} and \texttt{imtaQuestionReset}}
The \imtaInlinecode{latex}{\imtaQuestion} command outputs and formats a question counter.
It's meant to be used in reports for assignment with questions.
The counter should be reset with the \imtaInlinecode{latex}{\imtaQuestionReset}.
This couple of commands is meant to be used for sectioning when the assignment does not use a more explicit titling.

\subsection{\texttt{subsubsubsection}}
The \imtaInlinecode{bash}{imta_core} package offers a one-level-deeper section than the usual deepest \imtaInlinecode{latex}{\subsubsection}.
This provides an alternative to the usual \imtaInlinecode{latex}{\paragraph}.

\subsection{\texttt{chapter}}

\section{Document styling}

\subsection{IMT Atlantique styling}
\subsubsection{Colors}
The core package defines four colors, including the three colors of the IMT Atlantique, and a uniform and arbitrary gray.
These are defined as follows:

\begin{imtaCode}{latex}
\definecolor{imtaGreen}{RGB}{164, 210, 51}
\definecolor{imtaLightBlue}{RGB}{0, 184, 222}
\definecolor{imtaDarkBlue}{RGB}{12, 35, 64}
\definecolor{imtaGray}{RGB}{87, 87, 87}
\end{imtaCode}

Here are samples of these colors, with text in both black and white for previsualising the contrast.

\begin{figure}[H]
    \centering
    \resizebox{10cm}{!}{%
    \begin{tikzpicture}
        \fill[color=imtaGray] (0, 9) rectangle (6, 11);
        \fill[color=imtaGray] (7, 9) rectangle (13, 11);
        \node at (3, 10) {\LARGE \bf imtaGray};
        \node[white] at (10, 10) {\LARGE \bf imtaGray};

        \fill[color=imtaDarkBlue] (0, 6) rectangle (6, 8);
        \fill[color=imtaDarkBlue] (7, 6) rectangle (13, 8);
        \node at (3, 7) {\LARGE \bf imtaDarkBlue};
        \node[white] at (10, 7) {\LARGE \bf imtaDarkBlue};

        \fill[color=imtaLightBlue] (0, 3) rectangle (6, 5);
        \fill[color=imtaLightBlue] (7, 3) rectangle (13, 5);
        \node at (3, 4) {\LARGE \bf imtaLightBlue};
        \node[white] at (10, 4) {\LARGE \bf imtaLightBlue};

        \fill[color=imtaGreen] (0, 0) rectangle (6, 2);
        \fill[color=imtaGreen] (7, 0) rectangle (13, 2);
        \node at (3, 1) {\LARGE \bf imtaGreen};
        \node[white] at (10, 1) {\LARGE \bf imtaGreen};
    \end{tikzpicture}
    }
    \caption{Samples of the IMT Atlantique colors}
    \label{fig:imtaColors}
\end{figure}

\subsubsection{\texttt{imtaSetIMTStyle}}
\subsubsection{\texttt{imtaLogo}}
\subsubsection{\texttt{imtaLogoTikz}}
\subsubsection{\texttt{imtaMaketitlepage}}
\subsubsection{\texttt{imtaMakeCover}}

\section{External dependancies}

\subsection{\texttt{geometry}}
\begin{imtaCode}{latex}
\RequirePackage[a4paper, margin=2cm, top=3cm]{geometry}
\end{imtaCode}

\subsection{\texttt{graphicx}}
\begin{imtaCode}{latex}
\RequirePackage{graphicx}
\end{imtaCode}

\subsection{\texttt{fontenc}}
\begin{imtaCode}{latex}
\RequirePackage[T1]{fontenc}
\end{imtaCode}

\subsection{\texttt{hyperref}}
\begin{imtaCode}{latex}
\RequirePackage[hidelinks]{hyperref}
\end{imtaCode}

\subsection{\texttt{inputenc}}
\begin{imtaCode}{latex}
\RequirePackage[utf8]{inputenc}
\end{imtaCode}

\subsection{\texttt{fancyhdr}}
\begin{imtaCode}{latex}
\RequirePackage{fancyhdr}
\end{imtaCode}

\subsection{\texttt{tikz}}
\begin{imtaCode}{latex}
\RequirePackage{tikz}
\end{imtaCode}

\subsection{\texttt{titlesec}}
\begin{imtaCode}{latex}
\RequirePackage{titlesec}
\end{imtaCode}

\subsection{\texttt{titling}}
\begin{imtaCode}{latex}
\RequirePackage{titling}
\end{imtaCode}

\subsection{\texttt{sectsty}}
\begin{imtaCode}{latex}
\RequirePackage{sectsty}
\end{imtaCode}

\subsection{\texttt{etoolbox}}
\begin{imtaCode}{latex}
\RequirePackage{etoolbox}
\end{imtaCode}

\subsection{\texttt{hyphenat}}
\begin{imtaCode}{latex}
\RequirePackage[none]{hyphenat}
\end{imtaCode}

\subsection{\texttt{footmisc}}
\begin{imtaCode}{latex}
\RequirePackage[bottom]{footmisc}
\end{imtaCode}

% \chapter{Additional features: \texttt{imta_extra}}

\end{document}
%%%%%%%%%% END %%%%%%%%%% 
%%%%%%%%%%%%%%%%%%%%%%%%% 
